\documentclass{article}
\usepackage[utf8]{inputenc}
\usepackage[english, russian]{babel}
\usepackage[margin=0.5in]{geometry}
\usepackage{paralist}
\usepackage{amsthm, amsmath, amsfonts, amssymb}
\usepackage{mathtools} % \mathclap
\usepackage{bm}
\usepackage{dsfont}
\usepackage{hyperref}
\usepackage{tabularx}
\usepackage{graphicx}
\usepackage{multirow}
\usepackage{comment}
\usepackage{xcolor, colortbl}
\usepackage{soul}
\usepackage[normalem]{ulem}
\usepackage{xifthen, xspace}

\usepackage{sectsty}
\subsectionfont{\normalsize\color{red}}

\title{Домашнее задание №3 по курсу \\ <<Математическая Статистика в Машинном Обучении>>}
\author{Школа Анализа Данных}
\date{}

\renewenvironment{itemize}[1]{\begin{compactitem}#1}{\end{compactitem}}
\renewenvironment{enumerate}[1]{\begin{compactenum}#1}{\end{compactenum}}
\renewenvironment{description}[0]{\begin{compactdesc}}{\end{compactdesc}}


%%%%%%%%%%%%%%%%%%%%%%%%%%%%%%%%%%%%%%
%%%    Создание новых окружений   %%%%
%%%%%%%%%%%%%%%%%%%%%%%%%%%%%%%%%%%%%%

\theoremstyle{plain}
\newtheorem*{theorem*}{Теорема}
\newtheorem*{note-none}{Замечание}

\begin{document}

% Матрицы с вертикальными разделителями
\makeatletter
\renewcommand*\env@matrix[1][*\c@MaxMatrixCols c]{%
	\hskip -\arraycolsep
	\let\@ifnextchar\new@ifnextchar
	\array{#1}}
\makeatother

\newcommand{\TODO}[1]{\textbf{#1}}

% The list of general commands
\newcommand{\PI}{3.141592654}
\newcommand{\Sum}{\sum\limits}
\newcommand{\Int}{\int\limits}
\newcommand{\Lim}{\lim\limits}
\newcommand{\Prod}{\prod\limits}
\newcommand{\Intf}{\int\limits_{-\infty}^{+\infty}}
\newcommand{\Sumclap}[1]{\Sum_{\mathclap{#1}}}
\newcommand{\Intclap}[1]{\Int_{\mathclap{#1}}}
\newcommand{\Prodclap}[1]{\Prod_{\mathclap{#1}}}
\newcommand{\Aprod}{\bigodot}
\newcommand{\aprod}{\odot}
\newcommand{\Max}{\max\limits}
\newcommand{\Min}{\min\limits}
\newcommand{\argmax}{\arg\max}
\newcommand{\argmin}{\arg\min}


\newcommand{\lp}{\left(}
\newcommand{\rp}{\right)}
\newcommand{\lf}{\left\{}
\newcommand{\rf}{\right\}}
\newcommand{\ls}{\left[}
\newcommand{\rs}{\right]}
\newcommand{\lv}{\left|}
\newcommand{\rv}{\right|}
\newcommand{\la}{\left\langle}
\newcommand{\ra}{\right\rangle}

% Обозначения из предметной области: теории вероятностей и статистики
\newcommand{\Distr}{\mathsf{D}}
\newcommand{\Var}{\mathbb{V}}
\newcommand{\Exp}{\mathbb{E}}
\newcommand{\Cov}{\text{Cov}}
\newcommand{\Loss}{\mathcal{L}}
\newcommand{\loss}{\ell}
\newcommand{\LogLike}{\mathcal{L}}
\newcommand{\Like}{\ell}
\newcommand{\Risk}{\mathcal{R}}
\newcommand{\makebold}[1]{\boldsymbol{#1}}

\newcommand{\mean}[1]{\overline{#1}}
\newcommand{\avg}[1]{\left\langle #1 \right\rangle}
\newcommand{\angmean}[1]{\left\langle #1 \right\rangle}
\newcommand{\barmean}[1]{\overline{#1}}

\newcommand{\eps}{\varepsilon}
\renewcommand{\epsilon}{\varepsilon}
\newcommand{\Ind}{I}
\newcommand{\Fisher}{I}

\newcommand{\HOT}{\text{\textbf{H.O.T.}}}

\newcommand{\partfrac}[2]{\frac{\partial #1}{\partial #2}}
\newcommand{\ttt}[1]{\texttt{#1}}
\newcommand{\term}[1]{\textbf{#1}}


\renewcommand{\phi}{\varphi}

\newcommand{\CC}{\mathbb{C}}
\newcommand{\NN}{\mathbb{N}}
\newcommand{\PP}{\mathbb{R}}
\newcommand{\RR}{\mathbb{R}}
\newcommand{\XX}{\mathbb{X}}
\newcommand{\ZZ}{\mathbb{Z}}
\renewcommand{\AA}{\mathbb{A}}

\newcommand{\Xbb}{\mathbb{X}}
\newcommand{\Ybb}{\mathbb{Y}}
\newcommand{\Zbb}{\mathbb{Z}}

% Empirical values
\newcommand{\Ecdf}[1]{\hat{F}_n(#1)}
\newcommand{\OPT}{\ensuremath{\mathrm{OPT}\xspace}}
\newcommand{\opt}{\ensuremath{\mathrm{opt}\xspace}}
\newcommand{\boot}{\ensuremath{\mathrm{boot}\xspace}}
\newcommand{\bias}{\ensuremath{\mathrm{bias}\xspace}}
\newcommand{\se}{\ensuremath{\mathrm{se}\xspace}}
\newcommand{\MSE}{\ensuremath{\mathrm{MSE}\xspace}}
\newcommand{\RSS}{\ensuremath{\mathrm{RSS}\xspace}}
\newcommand{\qm}{\ensuremath{\mathrm{qm}\xspace}}
\newcommand{\as}{\ensuremath{\mathrm{as}\xspace}}
\newcommand{\trace}{\ensuremath{\mathrm{tr}\xspace}}
\newcommand{\const}{\ensuremath{\mathrm{const}\xspace}}
\newcommand{\sign}{\ensuremath{\mathrm{sign}\xspace}}
\newcommand{\tr}{\mathrm{tr}}
\newcommand{\new}{\mathrm{new}}
\newcommand{\lasso}{\mathrm{lasso}}
\newcommand{\old}{\mathrm{old}}
\newcommand{\diag}{\mathrm{diag}}
\newcommand{\rank}{\mathrm{rg}}
\newcommand{\ML}{\mathrm{ML}}
\newcommand{\MP}{\mathrm{MP}}
\newcommand{\KL}{\mathrm{KL}}
\newcommand{\NV}{\mathrm{NV}}
\newcommand{\MV}{\mathrm{MV}}
\newcommand{\LOO}{\mathrm{LOO}}
\newcommand{\IGMV}{\mathrm{IGMV}}
\newcommand{\MM}{\mathrm{MM}}
\newcommand{\nat}{\mathrm{nat}\xspace}
\newcommand{\grad}{\mathrm{grad}\xspace}
% Оценки
\newcommand{\esttheta}{\hat{\theta}}
\newcommand{\estlambda}{\hat{\lambda}}
\newcommand{\estmu}{\hat{\mu}}
\newcommand{\estsigma}{\hat{\sigma}}
\newcommand{\estalpha}{\hat{\alpha}}
\newcommand{\estbeta}{\hat{\beta}}
\newcommand{\estxi}{\hat{\xi}}
\newcommand{\esttau}{\hat{\tau}}
\newcommand{\estpsi}{\hat{\psi}}
\newcommand{\esta}{\hat{a}}
\newcommand{\estb}{\hat{b}}
\newcommand{\estc}{\hat{c}}
\newcommand{\estd}{\hat{d}}
\newcommand{\estf}{\hat{f}}
\newcommand{\estp}{\hat{p}}
\newcommand{\esty}{\hat{y}}
\newcommand{\estT}{\hat{T}}
\newcommand{\estR}{\hat{R}}
\newcommand{\estF}{\hat{F}}
\newcommand{\estC}{\hat{C}}
\newcommand{\estS}{\hat{S}}
\newcommand{\estY}{\hat{Y}}
\newcommand{\estVar}{\hat{\Var}}
\newcommand{\estExp}{\hat{\Exp}}
\newcommand{\estSe}{\hat{\se}}

\newcommand{\ecdf}{\hat{F}}

\newcommand{\hata}{\hat{a}}
\newcommand{\hatb}{\hat{b}}
\newcommand{\hatc}{\hat{c}}
\newcommand{\hatd}{\hat{d}}
\newcommand{\hatf}{\hat{f}}
\newcommand{\hatg}{\hat{g}}
\newcommand{\hatk}{\hat{k}}
\newcommand{\hatp}{\hat{p}}
\newcommand{\hatr}{\hat{r}}
\newcommand{\hatt}{\hat{t}}
\newcommand{\haty}{\hat{y}}
\newcommand{\hatw}{\hat{w}}

\newcommand{\hatC}{\hat{C}}
\newcommand{\hatF}{\hat{F}}
\newcommand{\hatJ}{\hat{J}}
\newcommand{\hatK}{\hat{K}}
\newcommand{\hatP}{\hat{P}}
\newcommand{\hatS}{\hat{S}}
\newcommand{\hatT}{\hat{T}}
\newcommand{\hatY}{\hat{Y}}
\newcommand{\hatV}{\hat{V}}
\newcommand{\hatU}{\hat{U}}


\newcommand{\hateps}{\hat{\eps}}
\newcommand{\hatalpha}{\hat{\alpha}}
\newcommand{\hatbeta}{\hat{\beta}}
\newcommand{\hatpsi}{\hat{\psi}}
\newcommand{\hatlambda}{\hat{\lambda}}
\newcommand{\hattheta}{\hat{\theta}}
\newcommand{\hatsigma}{\hat{\sigma}}
\newcommand{\hatmu}{\hat{\mu}}
\newcommand{\hatnu}{\hat{\nu}}
\newcommand{\hatSigma}{\hat{\Sigma}}
\newcommand{\hatSe}{\hat{\se}}
\newcommand{\hatExp}{\hat{\Exp}}
\newcommand{\hatVar}{\hat{\Var}}

\newcommand{\tilx}{\tilde{x}}
\newcommand{\tily}{\tilde{y}}
\newcommand{\tilX}{\tilde{X}}
\newcommand{\tilY}{\tilde{Y}}
\newcommand{\tilK}{\tilde{K}}
\newcommand{\tilU}{\tilde{U}}
\newcommand{\tilV}{\tilde{V}}
\newcommand{\tilSigma}{\tilde{\Sigma}}
\newcommand{\tiltau}{\tilde{\tau}}
\newcommand{\tiltheta}{\tilde{\theta}}
\newcommand{\tillambda}{\tilde{\lambda}}
\newcommand{\tilsigma}{\tilde{\sigma}}
\newcommand{\tilpsi}{\tilde{\psi}}
\newcommand{\tilmu}{\tilde{\mu}}

\newcommand{\MLE}{\text{MLE}}
\newcommand{\mlexi}{\xi_{MLE}}
\newcommand{\mletheta}{\theta_{MLE}}
\newcommand{\mlelambda}{\lambda_{MLE}}
\newcommand{\mlesigma}{\sigma_{MLE}}
\newcommand{\mlepsi}{\psi_{MLE}}
\newcommand{\mlemu}{\mu_{MLE}}
\newcommand{\mlenu}{\nu_{MLE}}

\newcommand{\mmxi}{\xi_{MM}}
\newcommand{\mmtheta}{\theta_{MM}}
\newcommand{\mmlambda}{\lambda_{MM}}
\newcommand{\mmsigma}{\sigma_{MM}}
\newcommand{\mmpsi}{\psi_{MM}}
\newcommand{\mmalpha}{\alpha_{MM}}
\newcommand{\mmbeta}{\beta_{MM}}

% Классы распределений
\newcommand{\Poisson}{\mathrm{Poisson}\xspace}
\newcommand{\Triangle}{\mathrm{Triangle}\xspace}
\newcommand{\Uniform}{\mathrm{Uniform}\xspace}
\newcommand{\Binomial}{\mathrm{Binomial}\xspace}
\newcommand{\Multinomial}{\mathrm{Multinomial}\xspace}
\newcommand{\Bernoulli}{\mathrm{Bernoulli}\xspace}
\newcommand{\Gammap}{\mathrm{Gamma}\xspace}
\newcommand{\Normal}{\mathcal{N}\xspace}
\newcommand{\LogN}{\mathrm{LogN}\xspace}
\newcommand{\Exponential}{\mathrm{Exp}\xspace}
\newcommand{\Erlang}{\mathrm{Erlang}\xspace}
\newcommand{\Cauchy}{C\xspace}
\newcommand{\Dir}{\mathrm{Dir}\xspace}
\newcommand{\Beta}{\mathrm{Beta}\xspace}
\newcommand{\Pareto}{\mathrm{Pareto}\xspace}
\newcommand{\Student}{\mathrm{Student}\xspace}
%
\newcommand{\Family}{\mathfrak{F}}

% Гипотезы
\newcommand{\RejectRegion}{R}
\newcommand{\pvalue}{\text{p-value}\xspace}
\newcommand{\llr}{\ell}
\newcommand{\Llr}{\mathcal{L}}

% Регрессия
\newcommand{\RRS}{\mathrm{RSS}\xspace}

\newcommand{\redtext}[1]{\textcolor{red}{#1}}
\newcommand{\addtask}[1]{\hyperref[#1]{\redtext{Задача~\ref*{#1}}}}
\newcommand{\solution}{\redtext{\textbf{Решение.}}}
\newcommand{\ignore}[1]{\xspace}


\newcommand{\NumOfSamples}{\mathcal{N}}
\newcommand{\NumOfDims}{\mathcal{D}}
\newcommand{\NumOfHidden}{\mathcal{H}}
\newcommand{\NumOfClasses}{\mathcal{K}}
\newcommand{\NumOfChannels}{\mathcal{C}}
\newcommand{\NumOfFilters}{\mathcal{F}}
\newcommand{\HiddenSize}{\mathcal{H}}

\newcommand{\boldzero}{\boldsymbol{0}}
\newcommand{\boldones}{\boldsymbol{1}}
\newcommand{\boldone}{\boldsymbol{1}}

\newcommand{\bolda}{\boldsymbol{a}}
\newcommand{\boldb}{\boldsymbol{b}}
\newcommand{\boldc}{\boldsymbol{c}}
\newcommand{\boldd}{\boldsymbol{d}}
\newcommand{\bolde}{\boldsymbol{e}}
\newcommand{\boldf}{\boldsymbol{f}}
\newcommand{\boldg}{\boldsymbol{g}}
\newcommand{\boldh}{\boldsymbol{h}}
\newcommand{\boldi}{\boldsymbol{i}}
\newcommand{\boldj}{\boldsymbol{j}}
\newcommand{\boldk}{\boldsymbol{k}}
\newcommand{\boldl}{\boldsymbol{l}}
\newcommand{\boldm}{\boldsymbol{m}}
\newcommand{\boldn}{\boldsymbol{n}}
\newcommand{\boldo}{\boldsymbol{o}}
\newcommand{\boldp}{\boldsymbol{p}}
\newcommand{\boldq}{\boldsymbol{q}}
\newcommand{\boldr}{\boldsymbol{r}}
\newcommand{\bolds}{\boldsymbol{s}}
\newcommand{\boldt}{\boldsymbol{t}}
\newcommand{\boldu}{\boldsymbol{u}}
\newcommand{\boldv}{\boldsymbol{v}}
\newcommand{\boldw}{\boldsymbol{w}}
\newcommand{\boldx}{\boldsymbol{x}}
\newcommand{\boldy}{\boldsymbol{y}}
\newcommand{\boldz}{\boldsymbol{z}}


\newcommand{\boldA}{\boldsymbol{A}}
\newcommand{\boldB}{\boldsymbol{B}}
\newcommand{\boldC}{\boldsymbol{C}}
\newcommand{\boldD}{\boldsymbol{D}}
\newcommand{\boldE}{\boldsymbol{E}}
\newcommand{\boldF}{\boldsymbol{F}}
\newcommand{\boldH}{\boldsymbol{H}}
\newcommand{\boldJ}{\boldsymbol{J}}
\newcommand{\boldK}{\boldsymbol{K}}
\newcommand{\boldL}{\boldsymbol{L}}
\newcommand{\boldM}{\boldsymbol{M}}
\newcommand{\boldN}{\boldsymbol{N}}
\newcommand{\boldI}{\boldsymbol{I}}
\newcommand{\boldP}{\boldsymbol{P}}
\newcommand{\boldQ}{\boldsymbol{Q}}
\newcommand{\boldR}{\boldsymbol{R}}
\newcommand{\boldS}{\boldsymbol{S}}
\newcommand{\boldT}{\boldsymbol{T}}
\newcommand{\boldO}{\boldsymbol{O}}
\newcommand{\boldU}{\boldsymbol{U}}
\newcommand{\boldV}{\boldsymbol{V}}
\newcommand{\boldW}{\boldsymbol{W}}
\newcommand{\boldX}{\boldsymbol{X}}
\newcommand{\boldY}{\boldsymbol{Y}}
\newcommand{\boldZ}{\boldsymbol{Z}}
\newcommand{\boldXY}{\boldsymbol{XY}}


\newcommand{\boldalpha}{\boldsymbol{\alpha}}
\newcommand{\boldbeta}{\boldsymbol{\beta}}
\newcommand{\boldtheta}{\boldsymbol{\theta}}
\newcommand{\boldmu}{\boldsymbol{\mu}}
\newcommand{\boldxi}{\boldsymbol{\xi}}
\newcommand{\boldeta}{\boldsymbol{\eta}}
\newcommand{\boldpi}{\boldsymbol{\pi}}
\newcommand{\boldsigma}{\boldsymbol{\sigma}}
\newcommand{\boldphi}{\boldsymbol{\phi}}
\newcommand{\boldpsi}{\boldsymbol{\psi}}
\newcommand{\boldlambda}{\boldsymbol{\lambda}}
\newcommand{\boldgamma}{\boldsymbol{\gamma}}
\newcommand{\bolddelta}{\boldsymbol{\delta}}
\newcommand{\boldeps}{\boldsymbol{\eps}}
\newcommand{\boldPhi}{\boldsymbol{\Phi}}
\newcommand{\boldPsi}{\boldsymbol{\Psi}}
\newcommand{\boldLambda}{\boldsymbol{\Lambda}}
\newcommand{\boldSigma}{\boldsymbol{\Sigma}}
\newcommand{\boldTheta}{\boldsymbol{\Theta}}
\newcommand{\boldOmega}{\boldsymbol{\Omega}}

\newcommand{\hatboldx}{\hat{\boldx}}
\newcommand{\hatboldk}{\hat{\boldk}}
\newcommand{\hatboldw}{\hat{\boldw}}
\newcommand{\hatboldp}{\hat{\boldp}}
\newcommand{\hatboldK}{\hat{\boldK}}
\newcommand{\hatboldC}{\hat{\boldC}}
\newcommand{\hatboldS}{\hat{\boldS}}
\newcommand{\hatboldU}{\hat{\boldU}}
\newcommand{\hatboldV}{\hat{\boldV}}
\newcommand{\hatboldX}{\hat{\boldX}}
\newcommand{\hatboldSigma}{\hat{\boldSigma}}
\newcommand{\hatboldLambda}{\hat{\boldLambda}}
\newcommand{\hatboldy}{\hat{\boldy}}
\newcommand{\hatboldmu}{\hat{\boldmu}}
\newcommand{\hatboldalpha}{\hat{\boldalpha}}
\newcommand{\hatboldbeta}{\hat{\boldbeta}}
\newcommand{\hatboldgamma}{\hat{\boldgamma}}
\newcommand{\hatboldtheta}{\hat{\bold\theta}}
\newcommand{\hatboldeps}{\hat{\boldeps}}
\newcommand{\hatbolddelta}{\hat{\bolddelta}}

\newcommand{\tilboldbeta}{\tilde{\boldbeta}}
\newcommand{\tilboldw}{\tilde{\boldw}}
\newcommand{\tilboldmu}{\tilde{\boldmu}}

\newcommand{\xs}[1]{\boldx^{#1}}
\newcommand{\ys}[1]{\boldy^{#1}}
\newcommand{\zs}[1]{\boldz^{#1}}
\newcommand{\Xs}[1]{\boldX^{#1}}
\newcommand{\Ys}[1]{\boldY^{#1}}
\newcommand{\Zs}[1]{\boldZ^{#1}}

\newcommand{\Ndim}{N}
\newcommand{\Ddim}{D}
\newcommand{\Mdim}{M}
\newcommand{\Kdim}{K}
\newcommand{\Adim}{A}
\newcommand{\Qdim}{Q}
\newcommand{\Rdim}{R}

\newcommand{\mcalA}{\mathcal{A}}
\newcommand{\mcalB}{\mathcal{B}}
\newcommand{\mcalC}{\mathcal{C}}
\newcommand{\mcalD}{\mathcal{D}}
\newcommand{\mcalE}{\mathcal{E}}
\newcommand{\mcalF}{\mathcal{F}}
\newcommand{\mcalI}{\mathcal{I}}
\newcommand{\mcalL}{\mathcal{L}}
\newcommand{\mcalP}{\mathcal{P}}
\newcommand{\mcalQ}{\mathcal{Q}}
\newcommand{\mcalX}{\mathcal{X}}
\newcommand{\hatmcalB}{\hat{\mcalB}}

\newcommand{\setA}{\mathcal{A}}
\newcommand{\setB}{\mathcal{B}}
\newcommand{\setC}{\mathcal{C}}
\newcommand{\setE}{\mathcal{E}}
\newcommand{\setD}{\mathcal{D}}
\newcommand{\setS}{\mathcal{S}}
\newcommand{\setT}{\mathcal{T}}
\newcommand{\setV}{\mathcal{V}}
\newcommand{\setW}{\mathcal{W}}

\newcommand{\matA}{A}
\newcommand{\matB}{B}
\newcommand{\matC}{C}
\newcommand{\matD}{D}
\newcommand{\matE}{E}
\newcommand{\matI}{I}
\newcommand{\matU}{U}
\newcommand{\matV}{V}
\newcommand{\matW}{W}
\newcommand{\matPhi}{\Phi}
\newcommand{\matPsi}{\Psi}


\newcommand{\Factors}{F}
\newcommand{\Variables}{X}
\newcommand{\Eye}{I}
\newcommand{\Zero}{O}
\newcommand{\Energy}{\mathcal{E}}
\newcommand{\Entropy}{\mathcal{H}}
\newcommand{\Fenergy}{F}
\newcommand{\Edges}{E}
\newcommand{\edge}{e}
\newcommand{\Vertices}{V}
\newcommand{\vertex}{v}
\newcommand{\Graph}{\mathcal{G}}
\newcommand{\Tree}{\mathcal{T}}
\newcommand{\Children}{\mathcal{C}}
\newcommand{\Parents}{\mathcal{P}}
\newcommand{\Adjacent}{\mathcal{A}}
\newcommand{\Pa}{\mathrm{Pa}}


\newcommand{\state}{z}
\newcommand{\State}{\boldz}
\newcommand{\StateR}{\boldZ}

\newcommand{\Covariance}{\Sigma}
\newcommand{\CovX}{\Covariance_{\boldX}}
\newcommand{\CovY}{\Covariance_{\boldY}}
\newcommand{\CovZ}{\Covariance_{\boldZ}}
\newcommand{\CovXY}{\Covariance_{\boldX\boldY}}

\newcommand{\hatCovariance}{\hat{\Covariance}}
\newcommand{\hatCovX}{\hatCovariance_{\boldX}}
\newcommand{\hatCovY}{\hatCovariance_{\boldY}}
\newcommand{\hatCovZ}{\hatCovariance_{\boldZ}}
\newcommand{\hatCovXY}{\hatCovariance_{\boldX\boldY}}

\newcommand{\tildeCovariance}{\tilde{\Covariance}}
\newcommand{\tildeCovX}{\tildeCovariance_{\boldX}}
\newcommand{\tildeCovY}{\tildeCovariance_{\boldY}}
\newcommand{\tildeCovZ}{\tildeCovariance_{\boldZ}}
\newcommand{\tildeCovXY}{\tildeCovariance_{\boldX\boldY}}


\newcommand{\hatState}{\hat{\State}}
\newcommand{\StateNum}{N}
\newcommand{\StateDim}{K}
\newcommand{\StateSet}{\ZZ}
\newcommand{\StatesSet}{\StateSet}
\newcommand{\NumStates}{N}
\newcommand{\StateToState}{A}
\newcommand{\StateCov}{\Sigma}
\newcommand{\StateJac}{A}

\newcommand{\hatStateCov}{\hat{\StateCov}}
\newcommand{\StateMean}{\boldmu}
\newcommand{\hatStateMean}{\hat{\StateMean}}
\newcommand{\StateToStateHistory}{\boldA}
\newcommand{\StateNoise}{\boldr}
\newcommand{\StateNoiseCov}{R}
\newcommand{\StateHistory}{\boldZ}
\newcommand{\StatesHistory}{\StateHistory}
\newcommand{\StateToObserv}{C}
\newcommand{\StateToobserv}{\boldc}
\newcommand{\StateToObservHistory}{\boldC}

\newcommand{\DState}{\bolddelta}
\newcommand{\hatDState}{\hat{\DState}}
\newcommand{\DStateMean}{\boldlambda}
\newcommand{\hatDStateMean}{\hat{\DStateMean}}
\newcommand{\DStateCov}{\Lambda}
\newcommand{\hatDStateCov}{\hat{\DStateCov}}

\newcommand{\DObserv}{\boldgamma}
\newcommand{\hatDObserv}{\hat{\DObserv}}

\newcommand{\observ}{x}
\newcommand{\Observ}{\boldsymbol{\observ}}
\newcommand{\ObservCov}{\Lambda}
\newcommand{\observMean}{\lambda}
\newcommand{\ObservMean}{\boldlambda}
\newcommand{\hatobserv}{\hat{\observ}}
\newcommand{\hatObserv}{\hat{\Observ}}
\newcommand{\hatObservCov}{\hat{\ObservCov}}
\newcommand{\hatobservMean}{\hat{\observMean}}
\newcommand{\hatObservMean}{\hat{\ObservMean}}

\newcommand{\ObservSet}{\XX}
\newcommand{\ObservNum}{N}
\newcommand{\ObservDim}{D}
\newcommand{\ObservSourceNum}{M}
\newcommand{\ObservHistory}{\boldX}
\newcommand{\ObservsHistory}{\ObservHistory}
\newcommand{\Timestamps}{\boldT}
\newcommand{\ObservJac}{H}
% Шум наблюдений
\newcommand{\observNoise}{q}
\newcommand{\ObservNoise}{\boldq}
\newcommand{\ObservNoiseCov}{Q}
\newcommand{\ObservNoiseCovHistory}{\boldQ}


\newcommand{\control}{u}
\newcommand{\Control}{\boldu}
\newcommand{\ControlNum}{N}
\newcommand{\ControToState}{B}
\newcommand{\ControlToStateHistory}{\boldB}
\newcommand{\ControlHistory}{\boldU}

\newcommand{\Jacobian}{\boldJ}

\newcommand{\Kalman}{K}
\newcommand{\kalman}{\boldk}

\newcommand{\vel}{v}
\maketitle

\section*{Задачи}
\subsection*{Теоретический блок}
\subsubsection*{Задача 1 [1 балл]}
Пусть дана обучающая выборка $\{(\boldX, \boldy) \colon \boldX \in \RR^{n \times d}, \boldy \in \RR^{n} \}$, $n \ge d$. Предположим, что справедлива следующая модель линейной регрессии:
$$
y = \boldx^T\boldw + \eps, \quad \eps \sim \Normal(0, \sigma^2),
$$
где $\boldw$ --- истинный, но неизвестный нам вектор весов. Пусть $\hatboldw$ --- MLE-оценка вектора весов $\boldw$.

Предположим, к нам поступили тестовые данные $\boldX^* \in \RR^{m \times d}$, для которых с помощью оценки $\hatboldw$ предсказываем вектор $\boldy^*\in \RR^{m}$.
Найдите математическое ожидание и матрицу ковариаций для вектора $\boldy^*$ (при условии фиксированной матрицы дизайна $\boldX$).

\subsubsection*{Задача 2 [1 балл]}
Пусть дана выборка $(\boldX,\boldt) = \{(\boldx_i, t_i)\colon \boldx_i \in \RR^d, t_i \in \RR\}_{i = 1}^n$, ($\boldX \in \RR^{n \times d}$, $\boldt \in \RR^n$, $n \ge d$). Предположим справедливость следующей модели данных
$$
t = \boldx^T\boldw + \eps(\boldx),
$$
где $\eps(\boldx) \sim \Normal(0, \sigma(\boldx)^2)$. Найдите MLE-оценку на вектор весов $\boldw$ в данном случае.

\subsubsection*{Задача 3 [2 балла]}
Пусть дана выборка $(\boldx, \boldy) = \{(x_i,y_i)\colon x_i, y_i \in \RR\}_{i=1}^n$. Пусть данные соответствуют модели
\begin{gather*}
	y_i = \beta x_i + \eps_i,
\end{gather*}
где $\epsilon_i \sim \Normal(0, \sigma^2)$. При этом значения $\boldx$ наблюдаются с ошибкой, т.е. представлена не выборка $(\boldx, \boldy)$, а выборка $(\boldz, \boldy) = \{(z_i, y_i)\colon z_i, y_i \in \RR\}_{i = 1}^n$, где $z_i = x_i + \delta_i$, $\delta_i \sim \Normal(0, \tau^2)$. Шумы $\eps_i$ и $\delta_i$ независимы. Оценим величину $\beta$, используя стандартный метод наименьших квадратов согласно формуле
\begin{gather*}
	\hatbeta = \frac{\sum_{i=1}^n z_i y_i}{\sum_{i=1}^n z_i^2}.
\end{gather*}
Докажите, что оценка $\hatbeta$ не является состоятельной. Для этого покажите, что $\hatbeta \xrightarrow[]{\Prob} a\beta$ при $n \to \infty$. Найдите явное выражение для $a$ в предположении, что точки $\{x_i\}_{i=1}^n$ поступают из некоторого распределения $F(x)$ с конечными первыми и вторыми моментами $\Exp(X)$ и $\Exp(X^2)$. 

\subsubsection*{Задача 4 [2 балла]}
Пусть дана обучающая выборка $\{(\boldx, \boldy) \colon \boldx \in \RR^{n}, \boldy \in \RR^{n} \}$. Предположим, что справедлива следующая модель линейной регрессии:
$$
y = w_0 + w_1 x + \eps,\qquad \eps \sim \Normal(0, \sigma^2).
$$
Сконструируйте асимтотический тест Вальда для проверки гипотезы $H_0 \colon w_1 = \alpha w_0$.

\textit{Внимание. Замечание про асимптотичность тут не просто так.}

\subsubsection*{Задача 5 [2 балла]}
Рассмотрим задачу восстановления регрессии. Модель регрессии имеет вид
$$
t = \boldx^T\boldw + \eps,
$$ 
где $\eps \sim \Normal(0,\beta^{-1})$, и на веса $\boldw$ наложено априорное распределение вида $p(\boldw) = \Normal(\boldw|\boldw_0,\boldS_0)$. 
Пусть дана выборка $(\boldX, \boldt) = \{(\boldx_i, t_i)\colon \boldx_i \in \RR^d, t_i \in \RR\}_{i=1}^n$. Найдите апостериорное распределение $p(\boldw|\boldX, \boldt)$.

\subsubsection*{Задача 6 [2 балла]}
Пусть $\boldx^{n} \sim f(\cdot)$, и пусть $\hatf(\cdot) = \hatf(\cdot;\boldx^n)$ обозначает ядерную оценку плотности на основе ядра
$$
K(x) = 
\begin{cases}
	1, &x \in \lp -\frac{1}{2}, \frac{1}{2} \rp;\\
	0, &\text{в противном случае}.
\end{cases}
$$
Найдите $\Exp [\hatf(x)]$ и $\Var [\hatf(x)]$. Покажите, что если $h\to0$ и $nh\to\infty$ при $n\to\infty$, то $\estf(x) \xrightarrow{\Prob} f(x)$ при $n\to\infty$.

\textit{Примечание. В ответе может быть использована истинная плотность $f(x)$.}

\subsubsection*{Задача 7 [4 балла]}
Рассмотрим задачу непараметрической оценки плотности распределения $p(x)$ по выборке $\boldX^{(N)}$. Обозначим через $\hatp(x;\boldX^{(N)})$
оценку плотности, полученную некоторым образом по выборке $\boldX^{(N)}$. Оценка риска для $\hatp(x;\boldX^{(N)})$ имеет вид:
$$
\hatJ(h) = \Int (\hatp(x;\boldX^{(N)}))^2 dx - \frac{2}{N} \Sum_{i=1}^N \hatp(X_i;\boldX^{(N\backslash i)}),
$$
где $\hatp(\cdot;\boldx^{(N\backslash i)})$ --- оценка плотности распределения на основе выборки $\boldX^{(N\backslash i)}$, т.е. выборки без объекта $X_i$.
\begin{itemize}
	\item (Гистограммная оценка) Разобьем диапазон наблюдаемых значений $\boldX^{(N)}$ на бины ширины $h$. Пусть в итоге значения $\boldX^{(n)}$ укладываются в $M$ последовательных бинов $B_1, \dots, B_M$. Пусть $N_m$ --- количество объектов выборки, попавших в $B_m$ ($\sum_m N_m = N$). Пусть $\hatp_m$ --- доля объектов выборки, попавших в бин $B_m$:
	$$
	N_m = \Sum_{i=1}^N I\ls X_i \in B_m\rs, \quad \hatp_m = \frac{N_m}{N}.
	$$
	
	Покажите, что в случае гистограммной оценки плотности оценка риска имеет вид:
	$$
	\hatJ(h) = \frac{2}{h(N - 1)} - \frac{N + 1}{h(N - 1)}\Sum_{m=1}^M \hatp_m^2.
	$$
	
	Докажите или опровергните равенство
	$$
	\Exp[\hatJ(h)] = \Exp[J(h)].
	$$
	Если равенство не верно, то чему равно $\Delta J(h) = \Exp[\hatJ(h)] - \Exp[J(h)]$?
	\item (Ядерная оценка) Покажите, что в случае ядерной оценки плотности оценка риска имеет вид:
	$$
	\hatJ(h) \approx \frac{1}{hN^2}\Sum_{i, j} K^*\lp \frac{X_i - X_j}{h}\rp + \frac{2}{N h}K(0),
	$$
	где $K^*(x) = K^{(2)}(x) - 2K(x)$ и $K^{(2)}(z) = \int K(z - y)K(y)dy$. В частности, если $K(x)$ --- это плотность нормального распределения $\Normal(0, 1)$, т.е. гауссово ядро, то $K^{(2)}(z)$ --- плотность распределения $\Normal(0, 2)$.
	
	Докажите или опровергните равенство
	$$
	\Exp[\hatJ(h)] =  \Exp[J(h)].
	$$
	Если равенство не верно, то чему равно $\Delta J(h) = \Exp[\hatJ(h)] - \Exp[J(h)]$?
\end{itemize}

\subsubsection*{Задача 8 [3 балла]}
Рассмотрим задачу непараметрической регрессии:
$$
Y_i = f(X_i) + \eps_i, \quad i \in \lf 1, \dots, n \rf, \quad X_i \in \RR, \quad Y_i \in \RR.
$$
где $\eps_i$ и $X_i$ независимы, $\Exp\eps_i = 0$, $\Var\eps_i = \sigma^2$, выборка $\{X_i\}_{i=1}^n$ одномерная и сэмплируется из отрезка $[0,1]$. Необходимо по имеющимся данным оценить функцию регрессии $f(x) = \Exp(Y|X=x)$.

\begin{enumerate}
	\item [a)] Рассмотрим следующее семейство функций
	$$
	\Family_M = \lf f(x) = \Sum_{i=1}^M c_i I[x\in B_i], c_i \in \RR, i=\overline{1, M} \rf, \text{ где } B_i = \ls\frac{i-1}{M}, \frac{i}{M}\rp.
	$$
	Последний отрезок $B_M$ включает обе граничные точки. Найдите функцию из класса $\Family_M$, которая минимизирует сумму квадратов ошибок:
	$$
	r(x;\boldX^n) = \argmin_{f(x) \in \Family_M}\Sum_{i=1}^n\lp Y_i - f(X_i)\rp^2
	$$
	\item [b)] Найдите функцию регрессии поточечно, решив в каждой точке $x$ следующую оптимизационную задачу:
	% Формула Надарая-Ватсона
	$$
	r(x;\boldX^n) = \argmin_{y \in \RR}\Sum_{i=1}^nK\lp\frac{x-X_i}{h}\rp\lp Y_i - y\rp^2,
	$$
	где $K(x)$ --- заданная ядерная функция, $h$ --- ширина ядра.
	\item [c)] Какая оценка получится, если изменить задачу на следующую:
	$$
	r(x;\boldX^n) = \argmin_{a, b \in \RR}\Sum_{i=1}^n K\lp\frac{x-X_i}{h}\rp\lp Y_i - a - bX_i\rp^2,
	$$
	где $K(x)$ --- заданная ядерная функция, $h$ --- ширина ядра?
\end{enumerate}

\subsection*{Практический блок}

\subsubsection*{Задача 9 [2 балла]}
Винни-Пуху на день рождения Сова подарила 5 горшочков с медом, каждый приблизительно весом 1 кг (исходя из объема горшочка и плотности мёда). Однако из проверенных источников (от Пятачка), Винни-Пух получил информацию, что один горшочек предположительно содержит неправильный мёд, причем его вес должен отличаться от 1 кг (из-за содержания неправильных веществ). Для проведения следственных мероприятий у ослика Иа-Иа были изъяты самодельные весы. Взвесив каждый горшочек индивидуально, Винни-Пух обнаружил, что весы явно имеют некоторую неизвестную погрешность взвешивания, так что проделанные измерения не позволяют однозначно проверить информацию о неправильности мёда в одном из горшочков. Поэтому Винни-Пух почему-то решил взвешивать горшочки сразу по два, но как только он закончил эти 10 взвешиваний, как ослик Иа-Иа, пригрозив судебными разбирательствами, в принудительном порядке затребовал свои весы обратно, оставив Винни-Пуха с результатами 15-и взвешиваний.

Нам даны результаты этих взвешиваний --- бинарная матрица $\boldX \in \{0, 1\}^{n \times d}$, где $n = 15$ и $d = 5$, и вектор $\boldy$ с результатами взвешиваний (\texttt{honey\_X.csv} и \texttt{honey\_y.csv}). По этим данных для каждого горшочка найдите p-value для гипотезы о том, что данный горшочек содержит неправильный мёд. Если ли среди горшочков такой, который на уровне значимости 95\% содержит неправильный мёд?

\textbf{Дополнительное задание на 1 балл}. Дисперсия веса горшочка зависит от дизайна взвешиваний (выбора матрицы $\boldX$). Возможно Винни-Пух ошибся, начав взвешивать горшочки сразу по два, и вместо этого стоило взвесить каждый горшочек отдельно от других ещё по два раза, получив в результате те же самые 15 взвешиваний до того момента, как весы были возвращены Иа-Иа. Найдите отношение дисперсии оценки веса горшочка в случае дизайна, предложенного Винни-Пухом, к дисперсии оценки веса горшочка в случае предложенного <<индивидуального>> дизайна. Какой дизайн лучше с точки зрения поиска горшочка с неправильным мёдом?

\subsubsection*{Задача 10 [3 балла]}
Скачайте данные \texttt{data.csv}, содержащие 12 столбцов независимых переменных и 1 столбец с зависимой переменной. Первые 250 строк отведите под обучение, а оставшиеся 1250 под тест (да, под обучение отводим сильно меньше).

\begin{itemize}
	\item Обучите простую линейную регрессию по обучающей выборке. Примените модель к тестовой выборке и найдите MSE.
	\item По обучающей выборке оцените наилучший набор признаков, описывающих выходную переменную. Используйте для этого статистику Cp Mallow, AIC-критерий, BIC-критерий, LOO-проверку. Выбор подмножества признаков проведите полным перебором. Позволяет ли какой-нибудь набор признаков получить значение MSE на тестовых данных меньше, чем на всех признаках?
	
	\textit{Внимание. В ответе должно быть понятно, какой набор признаков был выбран согласно каждому из критериев.}
\end{itemize}

\subsubsection*{Задача 11 [4 балла]}
Скачать данные со страницы курса (значения коэффициента преломления для разных типов стекла; первый столбец). Оценить плотность распределения этих значений, используя гистограмму и ядерную оценку. Для подбора ширины ячейки или ширины ядра использовать перекрестную проверку (кросс-проверку).  Для выбранных значений ширины ячейки и ширины ядра построить 95\%-ые доверительные интервалы для полученной оценки плотности.

\subsubsection*{Задача 12 [4 балла]}
По данным из предыдущей задачи, используя в качестве выходной переменной $y$ значения преломления для разных типов стекла, а в качестве входной переменной $x$ --- данные о содержании алюминия (четвертая переменная в матрице данных), восстановить зависимость между $y$ и $x$ с помощью ядерной непараметрической регрессии. Оценку ядра проводить с помощью перекрестной проверки. Построить 95\%-ые доверительные интервалы для полученной оценки функции регрессии.

\end{document}