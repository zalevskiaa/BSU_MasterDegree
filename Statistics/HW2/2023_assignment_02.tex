\documentclass{article}
\usepackage[utf8]{inputenc}
\usepackage[english, russian]{babel}
\usepackage[margin=0.5in]{geometry}
\usepackage{paralist}
\usepackage{amsthm, amsmath, amsfonts, amssymb}
\usepackage{mathtools} % \mathclap
\usepackage{bm}
\usepackage{dsfont}
\usepackage{hyperref}
\usepackage{tabularx}
\usepackage{graphicx}
\usepackage{multirow}
\usepackage{comment}
\usepackage{xcolor, colortbl}
\usepackage{xifthen, xspace}
\usepackage{caption, subcaption}

\usepackage{sectsty}
\subsectionfont{\normalsize\color{red}}

\title{Домашнее задание №2 по курсу \\ <<Математическая Статистика в Машинном Обучении>>}
\author{Школа Анализа Данных}
\date{}

\renewenvironment{itemize}[1]{\begin{compactitem}#1}{\end{compactitem}}
\renewenvironment{enumerate}[1]{\begin{compactenum}#1}{\end{compactenum}}
\renewenvironment{description}[0]{\begin{compactdesc}}{\end{compactdesc}}


\begin{document}

% Матрицы с вертикальными разделителями
\makeatletter
\renewcommand*\env@matrix[1][*\c@MaxMatrixCols c]{%
	\hskip -\arraycolsep
	\let\@ifnextchar\new@ifnextchar
	\array{#1}}
\makeatother

\newcommand{\TODO}[1]{\textbf{#1}}

% The list of general commands
\newcommand{\PI}{3.141592654}
\newcommand{\Sum}{\sum\limits}
\newcommand{\Int}{\int\limits}
\newcommand{\Lim}{\lim\limits}
\newcommand{\Prod}{\prod\limits}
\newcommand{\Intf}{\int\limits_{-\infty}^{+\infty}}
\newcommand{\Sumclap}[1]{\Sum_{\mathclap{#1}}}
\newcommand{\Intclap}[1]{\Int_{\mathclap{#1}}}
\newcommand{\Prodclap}[1]{\Prod_{\mathclap{#1}}}
\newcommand{\Aprod}{\bigodot}
\newcommand{\aprod}{\odot}
\newcommand{\Max}{\max\limits}
\newcommand{\Min}{\min\limits}
\newcommand{\argmax}{\arg\max}
\newcommand{\argmin}{\arg\min}


\newcommand{\lp}{\left(}
\newcommand{\rp}{\right)}
\newcommand{\lf}{\left\{}
\newcommand{\rf}{\right\}}
\newcommand{\ls}{\left[}
\newcommand{\rs}{\right]}
\newcommand{\lv}{\left|}
\newcommand{\rv}{\right|}
\newcommand{\la}{\left\langle}
\newcommand{\ra}{\right\rangle}

% Обозначения из предметной области: теории вероятностей и статистики
\newcommand{\Distr}{\mathsf{D}}
\newcommand{\Var}{\mathbb{V}}
\newcommand{\Exp}{\mathbb{E}}
\newcommand{\Cov}{\mathbb{C}\mathrm{ov}}
\newcommand{\Loss}{\mathcal{L}}
\newcommand{\loss}{\ell}
\newcommand{\LogLike}{\mathcal{L}}
\newcommand{\Like}{\ell}
\newcommand{\Risk}{\mathcal{R}}
\newcommand{\makebold}[1]{\boldsymbol{#1}}

\newcommand{\mean}[1]{\overline{#1}}
\newcommand{\avg}[1]{\overline{#1}}
\newcommand{\angmean}[1]{\left\langle #1 \right\rangle}
\newcommand{\barmean}[1]{\overline{#1}}

\newcommand{\eps}{\varepsilon}
\renewcommand{\epsilon}{\varepsilon}
\newcommand{\Ind}{I}
\newcommand{\Fisher}{I}

\newcommand{\HOT}{\text{\textbf{H.O.T.}}}

\newcommand{\partfrac}[2]{\frac{\partial #1}{\partial #2}}
\newcommand{\ttt}[1]{\texttt{#1}}
\newcommand{\term}[1]{\textbf{#1}}


\renewcommand{\phi}{\varphi}

\newcommand{\CC}{\mathbb{C}}
\newcommand{\NN}{\mathbb{N}}
\newcommand{\PP}{\mathbb{P}}
\newcommand{\RR}{\mathbb{R}}
\newcommand{\XX}{\mathbb{X}}
\newcommand{\ZZ}{\mathbb{Z}}
\renewcommand{\AA}{\mathbb{A}}

\newcommand{\Xbb}{\mathbb{X}}
\newcommand{\Ybb}{\mathbb{Y}}
\newcommand{\Zbb}{\mathbb{Z}}

% Empirical values
\newcommand{\Ecdf}[1]{\hat{F}_n(#1)}
\newcommand{\OPT}{\ensuremath{\mathrm{OPT}\xspace}}
\newcommand{\opt}{\ensuremath{\mathrm{opt}\xspace}}
\newcommand{\boot}{\ensuremath{\mathrm{boot}\xspace}}
\newcommand{\bias}{\ensuremath{\mathrm{bias}\xspace}}
\newcommand{\se}{\ensuremath{\mathrm{se}\xspace}}
\newcommand{\MSE}{\ensuremath{\mathrm{MSE}\xspace}}
\newcommand{\RSS}{\ensuremath{\mathrm{RSS}\xspace}}
\newcommand{\qm}{\ensuremath{\mathrm{qm}\xspace}}
\newcommand{\as}{\ensuremath{\mathrm{as}\xspace}}
\newcommand{\trace}{\ensuremath{\mathrm{tr}\xspace}}
\newcommand{\const}{\ensuremath{\mathrm{const}\xspace}}
\newcommand{\sign}{\ensuremath{\mathrm{sign}\xspace}}
\newcommand{\tr}{\mathrm{tr}}
\newcommand{\new}{\mathrm{new}}
\newcommand{\lasso}{\mathrm{lasso}}
\newcommand{\old}{\mathrm{old}}
\newcommand{\diag}{\mathrm{diag}}
\newcommand{\rank}{\mathrm{rg}}
\newcommand{\ML}{\mathrm{ML}}
\newcommand{\MP}{\mathrm{MP}}
\newcommand{\KL}{\mathrm{KL}}
\newcommand{\NV}{\mathrm{NV}}
\newcommand{\MV}{\mathrm{MV}}
\newcommand{\NP}{\mathrm{MP}}   % Нейман-Пирсон
\newcommand{\vs}{\mathrm{vs}}   % versus
\newcommand{\LOO}{\mathrm{LOO}}
\newcommand{\IGMV}{\mathrm{IGMV}}
\newcommand{\MM}{\mathrm{MM}}
\newcommand{\nat}{\mathrm{nat}\xspace}
\newcommand{\grad}{\mathrm{grad}\xspace}
% Оценки
\newcommand{\esttheta}{\hat{\theta}}
\newcommand{\estlambda}{\hat{\lambda}}
\newcommand{\estmu}{\hat{\mu}}
\newcommand{\estsigma}{\hat{\sigma}}
\newcommand{\estalpha}{\hat{\alpha}}
\newcommand{\estbeta}{\hat{\beta}}
\newcommand{\estxi}{\hat{\xi}}
\newcommand{\esttau}{\hat{\tau}}
\newcommand{\estpsi}{\hat{\psi}}
\newcommand{\esta}{\hat{a}}
\newcommand{\estb}{\hat{b}}
\newcommand{\estc}{\hat{c}}
\newcommand{\estd}{\hat{d}}
\newcommand{\estf}{\hat{f}}
\newcommand{\estp}{\hat{p}}
\newcommand{\esty}{\hat{y}}
\newcommand{\estT}{\hat{T}}
\newcommand{\estR}{\hat{R}}
\newcommand{\estF}{\hat{F}}
\newcommand{\estC}{\hat{C}}
\newcommand{\estS}{\hat{S}}
\newcommand{\estY}{\hat{Y}}
\newcommand{\estVar}{\hat{\Var}}
\newcommand{\estExp}{\hat{\Exp}}
\newcommand{\estSe}{\hat{\se}}

\newcommand{\ecdf}{\hat{F}}

\newcommand{\hata}{\hat{a}}
\newcommand{\hatb}{\hat{b}}
\newcommand{\hatc}{\hat{c}}
\newcommand{\hatd}{\hat{d}}
\newcommand{\hatf}{\hat{f}}
\newcommand{\hatg}{\hat{g}}
\newcommand{\hatk}{\hat{k}}
\newcommand{\hatp}{\hat{p}}
\newcommand{\hatr}{\hat{r}}
\newcommand{\hatt}{\hat{t}}
\newcommand{\haty}{\hat{y}}
\newcommand{\hatw}{\hat{w}}

\newcommand{\hatC}{\hat{C}}
\newcommand{\hatF}{\hat{F}}
\newcommand{\hatJ}{\hat{J}}
\newcommand{\hatK}{\hat{K}}
\newcommand{\hatP}{\hat{P}}
\newcommand{\hatS}{\hat{S}}
\newcommand{\hatT}{\hat{T}}
\newcommand{\hatY}{\hat{Y}}
\newcommand{\hatV}{\hat{V}}
\newcommand{\hatU}{\hat{U}}


\newcommand{\hateps}{\hat{\eps}}
\newcommand{\hatalpha}{\hat{\alpha}}
\newcommand{\hatbeta}{\hat{\beta}}
\newcommand{\hatpsi}{\hat{\psi}}
\newcommand{\hatlambda}{\hat{\lambda}}
\newcommand{\hattheta}{\hat{\theta}}
\newcommand{\hatsigma}{\hat{\sigma}}
\newcommand{\hatmu}{\hat{\mu}}
\newcommand{\hatnu}{\hat{\nu}}
\newcommand{\hatSigma}{\hat{\Sigma}}
\newcommand{\hatSe}{\hat{\se}}
\newcommand{\hatExp}{\hat{\Exp}}
\newcommand{\hatVar}{\hat{\Var}}

\newcommand{\tilx}{\tilde{x}}
\newcommand{\tily}{\tilde{y}}
\newcommand{\tilX}{\tilde{X}}
\newcommand{\tilY}{\tilde{Y}}
\newcommand{\tilK}{\tilde{K}}
\newcommand{\tilU}{\tilde{U}}
\newcommand{\tilV}{\tilde{V}}
\newcommand{\tilSigma}{\tilde{\Sigma}}
\newcommand{\tiltau}{\tilde{\tau}}
\newcommand{\tiltheta}{\tilde{\theta}}
\newcommand{\tillambda}{\tilde{\lambda}}
\newcommand{\tilsigma}{\tilde{\sigma}}
\newcommand{\tilpsi}{\tilde{\psi}}
\newcommand{\tilmu}{\tilde{\mu}}

\newcommand{\MLE}{\text{MLE}}
\newcommand{\mlexi}{\xi_{MLE}}
\newcommand{\mletheta}{\theta_{MLE}}
\newcommand{\mlelambda}{\lambda_{MLE}}
\newcommand{\mlesigma}{\sigma_{MLE}}
\newcommand{\mlepsi}{\psi_{MLE}}
\newcommand{\mlemu}{\mu_{MLE}}
\newcommand{\mlenu}{\nu_{MLE}}

\newcommand{\mmxi}{\xi_{MM}}
\newcommand{\mmtheta}{\theta_{MM}}
\newcommand{\mmlambda}{\lambda_{MM}}
\newcommand{\mmsigma}{\sigma_{MM}}
\newcommand{\mmpsi}{\psi_{MM}}
\newcommand{\mmalpha}{\alpha_{MM}}
\newcommand{\mmbeta}{\beta_{MM}}

% Классы распределений
\newcommand{\Poisson}{\mathrm{Poisson}\xspace}
\newcommand{\Triangle}{\mathrm{Triangle}\xspace}
\newcommand{\Uniform}{\mathrm{Uniform}\xspace}
\newcommand{\Binomial}{\mathrm{Binomial}\xspace}
\newcommand{\Multinomial}{\mathrm{Multinomial}\xspace}
\newcommand{\Bernoulli}{\mathrm{Bernoulli}\xspace}
\newcommand{\Gammap}{\mathrm{Gamma}\xspace}
\newcommand{\Normal}{\mathcal{N}\xspace}
\newcommand{\Student}{\mathcal{T}\xspace}
%\newcommand{\Student}{\mathrm{Student}\xspace}
\newcommand{\LogN}{\mathrm{LogN}\xspace}
\newcommand{\Exponential}{\mathrm{Exp}\xspace}
\newcommand{\Erlang}{\mathrm{Erlang}\xspace}
\newcommand{\Cauchy}{C\xspace}
\newcommand{\Dir}{\mathrm{Dir}\xspace}
\newcommand{\Beta}{\mathrm{Beta}\xspace}
\newcommand{\Pareto}{\mathrm{Pareto}\xspace}

%
\newcommand{\Family}{\mathfrak{F}}

% Гипотезы
\newcommand{\RejectRegion}{R}
\newcommand{\pvalue}{\text{p-value}\xspace}
\newcommand{\llr}{\ell}
\newcommand{\Llr}{\mathcal{L}}

% Регрессия
\newcommand{\RRS}{\mathrm{RSS}\xspace}

\newcommand{\redtext}[1]{\textcolor{red}{#1}}
\newcommand{\addtask}[1]{\hyperref[#1]{\redtext{Задача~\ref*{#1}}}}
\newcommand{\solution}{\redtext{\textbf{Решение.}}}
\newcommand{\ignore}[1]{\xspace}


\newcommand{\NumOfSamples}{\mathcal{N}}
\newcommand{\NumOfDims}{\mathcal{D}}
\newcommand{\NumOfHidden}{\mathcal{H}}
\newcommand{\NumOfClasses}{\mathcal{K}}
\newcommand{\NumOfChannels}{\mathcal{C}}
\newcommand{\NumOfFilters}{\mathcal{F}}
\newcommand{\HiddenSize}{\mathcal{H}}

\newcommand{\boldzero}{\boldsymbol{0}}
\newcommand{\boldones}{\boldsymbol{1}}
\newcommand{\boldone}{\boldsymbol{1}}

\newcommand{\bolda}{\boldsymbol{a}}
\newcommand{\boldb}{\boldsymbol{b}}
\newcommand{\boldc}{\boldsymbol{c}}
\newcommand{\boldd}{\boldsymbol{d}}
\newcommand{\bolde}{\boldsymbol{e}}
\newcommand{\boldf}{\boldsymbol{f}}
\newcommand{\boldg}{\boldsymbol{g}}
\newcommand{\boldh}{\boldsymbol{h}}
\newcommand{\boldi}{\boldsymbol{i}}
\newcommand{\boldj}{\boldsymbol{j}}
\newcommand{\boldk}{\boldsymbol{k}}
\newcommand{\boldl}{\boldsymbol{l}}
\newcommand{\boldm}{\boldsymbol{m}}
\newcommand{\boldn}{\boldsymbol{n}}
\newcommand{\boldo}{\boldsymbol{o}}
\newcommand{\boldp}{\boldsymbol{p}}
\newcommand{\boldq}{\boldsymbol{q}}
\newcommand{\boldr}{\boldsymbol{r}}
\newcommand{\bolds}{\boldsymbol{s}}
\newcommand{\boldt}{\boldsymbol{t}}
\newcommand{\boldu}{\boldsymbol{u}}
\newcommand{\boldv}{\boldsymbol{v}}
\newcommand{\boldw}{\boldsymbol{w}}
\newcommand{\boldx}{\boldsymbol{x}}
\newcommand{\boldy}{\boldsymbol{y}}
\newcommand{\boldz}{\boldsymbol{z}}


\newcommand{\boldA}{\boldsymbol{A}}
\newcommand{\boldB}{\boldsymbol{B}}
\newcommand{\boldC}{\boldsymbol{C}}
\newcommand{\boldD}{\boldsymbol{D}}
\newcommand{\boldE}{\boldsymbol{E}}
\newcommand{\boldF}{\boldsymbol{F}}
\newcommand{\boldH}{\boldsymbol{H}}
\newcommand{\boldJ}{\boldsymbol{J}}
\newcommand{\boldK}{\boldsymbol{K}}
\newcommand{\boldL}{\boldsymbol{L}}
\newcommand{\boldM}{\boldsymbol{M}}
\newcommand{\boldN}{\boldsymbol{N}}
\newcommand{\boldI}{\boldsymbol{I}}
\newcommand{\boldP}{\boldsymbol{P}}
\newcommand{\boldQ}{\boldsymbol{Q}}
\newcommand{\boldR}{\boldsymbol{R}}
\newcommand{\boldS}{\boldsymbol{S}}
\newcommand{\boldT}{\boldsymbol{T}}
\newcommand{\boldO}{\boldsymbol{O}}
\newcommand{\boldU}{\boldsymbol{U}}
\newcommand{\boldV}{\boldsymbol{V}}
\newcommand{\boldW}{\boldsymbol{W}}
\newcommand{\boldX}{\boldsymbol{X}}
\newcommand{\boldY}{\boldsymbol{Y}}
\newcommand{\boldZ}{\boldsymbol{Z}}
\newcommand{\boldXY}{\boldsymbol{XY}}


\newcommand{\boldalpha}{\boldsymbol{\alpha}}
\newcommand{\boldbeta}{\boldsymbol{\beta}}
\newcommand{\boldtheta}{\boldsymbol{\theta}}
\newcommand{\boldmu}{\boldsymbol{\mu}}
\newcommand{\boldxi}{\boldsymbol{\xi}}
\newcommand{\boldeta}{\boldsymbol{\eta}}
\newcommand{\boldpi}{\boldsymbol{\pi}}
\newcommand{\boldsigma}{\boldsymbol{\sigma}}
\newcommand{\boldphi}{\boldsymbol{\phi}}
\newcommand{\boldpsi}{\boldsymbol{\psi}}
\newcommand{\boldlambda}{\boldsymbol{\lambda}}
\newcommand{\boldgamma}{\boldsymbol{\gamma}}
\newcommand{\bolddelta}{\boldsymbol{\delta}}
\newcommand{\boldeps}{\boldsymbol{\eps}}
\newcommand{\boldPhi}{\boldsymbol{\Phi}}
\newcommand{\boldPsi}{\boldsymbol{\Psi}}
\newcommand{\boldLambda}{\boldsymbol{\Lambda}}
\newcommand{\boldSigma}{\boldsymbol{\Sigma}}
\newcommand{\boldTheta}{\boldsymbol{\Theta}}
\newcommand{\boldOmega}{\boldsymbol{\Omega}}

\newcommand{\hatboldx}{\hat{\boldx}}
\newcommand{\hatboldk}{\hat{\boldk}}
\newcommand{\hatboldw}{\hat{\boldw}}
\newcommand{\hatboldp}{\hat{\boldp}}
\newcommand{\hatboldK}{\hat{\boldK}}
\newcommand{\hatboldC}{\hat{\boldC}}
\newcommand{\hatboldS}{\hat{\boldS}}
\newcommand{\hatboldU}{\hat{\boldU}}
\newcommand{\hatboldV}{\hat{\boldV}}
\newcommand{\hatboldX}{\hat{\boldX}}
\newcommand{\hatboldSigma}{\hat{\boldSigma}}
\newcommand{\hatboldLambda}{\hat{\boldLambda}}
\newcommand{\hatboldy}{\hat{\boldy}}
\newcommand{\hatboldmu}{\hat{\boldmu}}
\newcommand{\hatboldalpha}{\hat{\boldalpha}}
\newcommand{\hatboldbeta}{\hat{\boldbeta}}
\newcommand{\hatboldgamma}{\hat{\boldgamma}}
\newcommand{\hatboldtheta}{\hat{\bold\theta}}
\newcommand{\hatboldeps}{\hat{\boldeps}}
\newcommand{\hatbolddelta}{\hat{\bolddelta}}

\newcommand{\tilboldbeta}{\tilde{\boldbeta}}
\newcommand{\tilboldw}{\tilde{\boldw}}
\newcommand{\tilboldmu}{\tilde{\boldmu}}

\newcommand{\xs}[1]{\boldx^{#1}}
\newcommand{\ys}[1]{\boldy^{#1}}
\newcommand{\zs}[1]{\boldz^{#1}}
\newcommand{\Xs}[1]{\boldX^{#1}}
\newcommand{\Ys}[1]{\boldY^{#1}}
\newcommand{\Zs}[1]{\boldZ^{#1}}

\newcommand{\Ndim}{N}
\newcommand{\Ddim}{D}
\newcommand{\Mdim}{M}
\newcommand{\Kdim}{K}
\newcommand{\Adim}{A}
\newcommand{\Qdim}{Q}
\newcommand{\Rdim}{R}

\newcommand{\mcalA}{\mathcal{A}}
\newcommand{\mcalB}{\mathcal{B}}
\newcommand{\mcalC}{\mathcal{C}}
\newcommand{\mcalD}{\mathcal{D}}
\newcommand{\mcalE}{\mathcal{E}}
\newcommand{\mcalF}{\mathcal{F}}
\newcommand{\mcalI}{\mathcal{I}}
\newcommand{\mcalL}{\mathcal{L}}
\newcommand{\mcalP}{\mathcal{P}}
\newcommand{\mcalQ}{\mathcal{Q}}
\newcommand{\mcalX}{\mathcal{X}}
\newcommand{\hatmcalB}{\hat{\mcalB}}

\newcommand{\setA}{\mathcal{A}}
\newcommand{\setB}{\mathcal{B}}
\newcommand{\setC}{\mathcal{C}}
\newcommand{\setE}{\mathcal{E}}
\newcommand{\setD}{\mathcal{D}}
\newcommand{\setS}{\mathcal{S}}
\newcommand{\setT}{\mathcal{T}}
\newcommand{\setV}{\mathcal{V}}
\newcommand{\setW}{\mathcal{W}}

\newcommand{\matA}{A}
\newcommand{\matB}{B}
\newcommand{\matC}{C}
\newcommand{\matD}{D}
\newcommand{\matE}{E}
\newcommand{\matI}{I}
\newcommand{\matU}{U}
\newcommand{\matV}{V}
\newcommand{\matW}{W}
\newcommand{\matPhi}{\Phi}
\newcommand{\matPsi}{\Psi}


\newcommand{\Factors}{F}
\newcommand{\Variables}{X}
\newcommand{\Eye}{I}
\newcommand{\Zero}{O}
\newcommand{\Energy}{\mathcal{E}}
\newcommand{\Entropy}{\mathcal{H}}
\newcommand{\Fenergy}{F}
\newcommand{\Edges}{E}
\newcommand{\edge}{e}
\newcommand{\Vertices}{V}
\newcommand{\vertex}{v}
\newcommand{\Graph}{\mathcal{G}}
\newcommand{\Tree}{\mathcal{T}}
\newcommand{\Children}{\mathcal{C}}
\newcommand{\Parents}{\mathcal{P}}
\newcommand{\Adjacent}{\mathcal{A}}
\newcommand{\Pa}{\mathrm{Pa}}


\newcommand{\state}{z}
\newcommand{\State}{\boldz}
\newcommand{\StateR}{\boldZ}

\newcommand{\Covariance}{\Sigma}
\newcommand{\CovX}{\Covariance_{\boldX}}
\newcommand{\CovY}{\Covariance_{\boldY}}
\newcommand{\CovZ}{\Covariance_{\boldZ}}
\newcommand{\CovXY}{\Covariance_{\boldX\boldY}}

\newcommand{\hatCovariance}{\hat{\Covariance}}
\newcommand{\hatCovX}{\hatCovariance_{\boldX}}
\newcommand{\hatCovY}{\hatCovariance_{\boldY}}
\newcommand{\hatCovZ}{\hatCovariance_{\boldZ}}
\newcommand{\hatCovXY}{\hatCovariance_{\boldX\boldY}}

\newcommand{\tildeCovariance}{\tilde{\Covariance}}
\newcommand{\tildeCovX}{\tildeCovariance_{\boldX}}
\newcommand{\tildeCovY}{\tildeCovariance_{\boldY}}
\newcommand{\tildeCovZ}{\tildeCovariance_{\boldZ}}
\newcommand{\tildeCovXY}{\tildeCovariance_{\boldX\boldY}}


\newcommand{\hatState}{\hat{\State}}
\newcommand{\StateNum}{N}
\newcommand{\StateDim}{K}
\newcommand{\StateSet}{\ZZ}
\newcommand{\StatesSet}{\StateSet}
\newcommand{\NumStates}{N}
\newcommand{\StateToState}{A}
\newcommand{\StateCov}{\Sigma}
\newcommand{\StateJac}{A}

\newcommand{\hatStateCov}{\hat{\StateCov}}
\newcommand{\StateMean}{\boldmu}
\newcommand{\hatStateMean}{\hat{\StateMean}}
\newcommand{\StateToStateHistory}{\boldA}
\newcommand{\StateNoise}{\boldr}
\newcommand{\StateNoiseCov}{R}
\newcommand{\StateHistory}{\boldZ}
\newcommand{\StatesHistory}{\StateHistory}
\newcommand{\StateToObserv}{C}
\newcommand{\StateToobserv}{\boldc}
\newcommand{\StateToObservHistory}{\boldC}

\newcommand{\DState}{\bolddelta}
\newcommand{\hatDState}{\hat{\DState}}
\newcommand{\DStateMean}{\boldlambda}
\newcommand{\hatDStateMean}{\hat{\DStateMean}}
\newcommand{\DStateCov}{\Lambda}
\newcommand{\hatDStateCov}{\hat{\DStateCov}}

\newcommand{\DObserv}{\boldgamma}
\newcommand{\hatDObserv}{\hat{\DObserv}}

\newcommand{\observ}{x}
\newcommand{\Observ}{\boldsymbol{\observ}}
\newcommand{\ObservCov}{\Lambda}
\newcommand{\observMean}{\lambda}
\newcommand{\ObservMean}{\boldlambda}
\newcommand{\hatobserv}{\hat{\observ}}
\newcommand{\hatObserv}{\hat{\Observ}}
\newcommand{\hatObservCov}{\hat{\ObservCov}}
\newcommand{\hatobservMean}{\hat{\observMean}}
\newcommand{\hatObservMean}{\hat{\ObservMean}}

\newcommand{\ObservSet}{\XX}
\newcommand{\ObservNum}{N}
\newcommand{\ObservDim}{D}
\newcommand{\ObservSourceNum}{M}
\newcommand{\ObservHistory}{\boldX}
\newcommand{\ObservsHistory}{\ObservHistory}
\newcommand{\Timestamps}{\boldT}
\newcommand{\ObservJac}{H}
% Шум наблюдений
\newcommand{\observNoise}{q}
\newcommand{\ObservNoise}{\boldq}
\newcommand{\ObservNoiseCov}{Q}
\newcommand{\ObservNoiseCovHistory}{\boldQ}


\newcommand{\control}{u}
\newcommand{\Control}{\boldu}
\newcommand{\ControlNum}{N}
\newcommand{\ControToState}{B}
\newcommand{\ControlToStateHistory}{\boldB}
\newcommand{\ControlHistory}{\boldU}

\newcommand{\Jacobian}{\boldJ}

\newcommand{\Kalman}{K}
\newcommand{\kalman}{\boldk}

\newcommand{\vel}{v}
\maketitle



\subsection*{Задачи}
\subsubsection*{Задача 1 [4 балла]}
Пусть $n_1$ --- количество людей, которые получили лечение по методике 1, а $n_2$ --- количество людей, которые получили лечение по методике 2. Обозначим через $X_1$ --- количество людей, получивших лечение по методике 1, на которых эта методика повлияла положительно. Аналогично, обозначим через $X_2$ --- количество людей, получивших лечение по методике 2, на которых эта методика повлияла положительно. Предположим, что $X_1 \sim \Binomial(n_1,p_1)$ и $X_2\sim \Binomial(n_2,p_2)$. Положим $\psi = p_1-p_2$.
\begin{itemize}
	\item[(a)] Найдите MLE-оценку $\mlepsi$ для параметра $\psi$.
	\item[(b)] Найдите информационную матрицу Фишера $I(p_1,p_2)$.
	\item[(c)] Используя многопараметрический дельта-метод найдите асимптотическую стандартную ошибку для $\mlepsi$.
	\item[(d)] Допустим, что $n_1=n_2=200$, и конкретные значения случайных величин $X_1$ и $X_2$ равны $160$ и $148$ соответственно. Чему в этом случае равна оценка  $\mlepsi$. Найдите приблизительный (асимптотический) 90\%-ый доверительный интервал для $\psi$, используя (а) многопараметрический дельта-метод и (б) параметрический бутстреп.
\end{itemize}

\subsubsection*{Задача 2 [2 балла]}
Пусть $\boldX =\{X_1,\ldots,X_n\} \sim \Poisson(\lambda)$.
\begin{itemize}
	\item Постройте оценки $\tillambda$ параметра $\lambda$ с помощью метода моментов с использованием пробных функций $g_1(x) = x$ и $g_2(x) = x^2$.
	\item Постройте оценку $\hatlambda$ параметра $\lambda$ с помощью метода максимального правдоподобия. Найдите информацию Фишера $I_{X}(\lambda)$. Является ли оценка $\hatlambda$ эффективной?
\end{itemize}

\subsubsection*{Задача 3 [4 балла]}
Пусть $\boldX =\{X_1,\dots,X_n\} \sim \Pareto(\theta, \nu)$, $\theta > 0$, $\nu > 0$, с функцией плотности
$$
f_{\theta, \nu}(x) =
\begin{cases}
	\frac{\theta\nu^{\theta}}{x^{\theta+1}}, \quad & x \ge \nu,\\
	0, \quad & x < \nu
\end{cases}
$$
\begin{itemize}
	\item[a)] Найдите MLE-оценки $\hattheta$ и $\hat \nu$ для параметров $\theta$ и $\nu$.
	\item[c)] Пусть параметр $\nu$ известен. Найдите истинные значения $\Exp_{\theta}[\hat{\theta}]$ и $\Var_{\theta}[\hat{\theta}]$ как функции параметров $\theta$, $\nu$ и размера выборки $n$.
	\textit{Подсказка: следует использовать тот факт, что логарифм от случайной величины с распределением Парето, имеет экспоненциальное распределение.}
	\item[b)] Пусть параметр $\nu$ известен. Найдите асимптотическое распределение оценки $\hattheta$ с помощью дельта-метода.
	\item[d)] Пусть параметр $\nu$ известен. Найдите информацию Фишера $I_X(\theta)$. Является ли MLE-оценка параметра $\hat{\theta}$ эффективной?
\end{itemize}

\subsubsection*{Задача 4 [4 балла]}
Пусть $\boldX = \{X_1,\ldots,X_n\} \sim \Uniform(0,\theta)$, $Y = \max\{X_1,\ldots,X_n\}$.  Необходимо протестировать основную гипотезу $H_0:\theta = 1/2$ против альтернативы $H_1: \theta > 1/2$. В данном случае нельзя использовать тест Вальда, так как $Y$ при $n\to\infty$ не сходится к нормальному распределению. Допустим, что мы будем использовать следующее правило: гипотеза $H_0$ отвергается, если $Y > c$. 
\begin{itemize}
	\item[(a)] Найдите функцию мощности для данного теста.
	\item[(b)] При каком значении параметра $c$ размер теста будет равен $0.05$?
	\item[(c)] Каково значение $\pvalue$, если размер выборки $n = 20$ и $Y = 0.48$? Что можно сказать о гипотезе $H_0$?
	\item[(d)] Каково значение $\pvalue$, если размер выборки $n = 20$ и $Y = 0.52$? Что можно сказать о гипотезе $H_0$?
\end{itemize}

\subsubsection*{Задача 5 [1 балл]}
Пусть $\boldX = \{X_1,\dots,X_n\} \sim \Exponential(\theta)$. Постройте критерий отношения правдоподобий для проверки гипотезы $H_0: \theta = \theta_0$ vs $H_1 : \theta > \theta_0$.

\subsubsection*{Задача 6 [3 балла]}
Пусть $\boldX = \{X_1,\ldots,X_n\} \sim \Normal(\mu,\sigma^2)$, где параметр $\mu$ известен. Требуется протестировать гипотезу $H_0\colon \sigma = \sigma_0$ против альтернативы $H_1 \colon \sigma \neq \sigma_0$.
\begin{itemize}
	\item Постройте критерий отношения правдоподобий для различения гипотез $H_0$ и $H_1$.
	\item Постройте критерий Вальда для различения гипотез $H_0$ и $H_1$.
	\item Сравните аналитически полученные критерии.
	%Сравнить (как аналитически, так и экспериментально) полученный тест с тестом Вальда для различения этих гипотез.
\end{itemize}

\textit{Примечание.} Аналитическое сравнение тестов подразумевает доказательство их (асимптотической) эквивалентности или неэквивалентности, где под эквивалентностью понимается идентичность выносимых тестами решений.


\subsubsection*{Задача 7 [2 балла]}
Пусть $\boldX = \{X_1,\dots,X_n\}$ --- выборка н.о.р. с.в. со следующей функцией плотности:
$$
f(x, \theta) = \begin{cases}
	c(\theta)d(x), &a \leqslant x \leqslant b(\theta) \\
	0, &\text{ иначе }
\end{cases}
$$
где $b(\theta)$ --- монотонно возрастающая функция одного аргумента.

\begin{enumerate}
	\item[(a)] Построить статистику отношения правдоподобий $\lambda$ для тестирования гипотезы $H_0: \theta = \theta_0$ vs $H_1: \theta \neq \theta_0$
	\item[(b)] Найти распределение статистики $\lambda$ при выполнении $H_0$ для следующей функции плотности:
	$$
	f(x, \theta) = \begin{cases}
		\frac{2x}{\theta^2}, \quad &0 \leqslant x \leqslant \theta \\
		0, \quad &\text{иначе}
	\end{cases}
	$$
\end{enumerate}

\subsubsection*{Задача 8 [2 балла]}
Найдите наилучшую критическую область (НКО) для проверки гипотезы $H_0 \colon \Uniform[-a, a]$ против гипотезы $H_1 \colon \Normal(0, \sigma^2)$ по одному
наблюдению $(n = 1)$ при уровне значимости $\alpha = 0.1$. Найдите мощность полученного критерия.

\subsubsection*{Задача 9 [2 балла]}
Проверяются гипотезы о плотности $f$ распределения наблюдений $\boldX = \{X_1,\dots,X_n\}$: гипотеза $H_0\colon f = f_0$ против альтернативы $H_1\colon f = f_1$, где
\begin{gather*}
	f_1(x) = 
	\begin{cases}
		1, &x \in [0,1],\\
		0, &x \notin [0, 1],
	\end{cases}
	\qquad
	f_2(x)=
	\begin{cases}
		2x, &x \in [0, 1], \\
		0, &x \notin [0, 1].
	\end{cases}
\end{gather*}
Построить наиболее мощный критерий размера $\alpha$ при $n = 1$ и $n = 2$.

\subsubsection*{Задача 10 [2 балла]}
В процессе настольной игры у игроков возникло подозрение, что два кубика, которые шли в комплекте с игрой, несимметричны. Поэтому, начиная с некоторого момента, они начали записывать результаты бросков. В каждом броске участвуют оба кубика. Результаты приведены в таблице.
\begin{table}[h]
	\centering
	\begin{tabular}{|l|c|c|c|c|c|c|c|c|c|c|c|}
		\hline
		Сумма очков & 2 & 3 & 4 & 5 & 6 & 7 & 8 & 9 & 10 & 11 & 12\\ \hline
		Количество бросков & 2 & 4 & 20 & 18 & 34 & 41 & 32 & 26 & 16 & 9 & 12 \\
		\hline
	\end{tabular}
\end{table}

Проверьте гипотезу о том, что оба кубика симметричны на уровне значимости $\alpha = 0.05$. Найдите p-value.

\subsubsection*{Задача 11 [2 балла]}
Предположим, что у нас есть 10 статей, написанных автором, скрывающемся под псевдонимом. Мы подозреваем, что эти статьи на самом деле написаны некоторым известным писателем. Чтобы проверить эту гипотезу, мы подсчитали доли четырехбуквенных слов в 8-и сочинениях подозреваемого нами автора:
$$
.224~ .261~ .216~ .239~ .229~ .228~ .234~ .216~
$$
В 10 сочинениях, опубликованных под псевдонимом, доли четырехбуквенных слов равны
$$
.207~ .204~ .195~ .209~ .201~ .206~ .223~ .222~ .219~ .200
$$
\begin{itemize}
	\item Используйте критерий Вальда. Найдите $\pvalue$ и 95\%-ый доверительный интервал для разницы средних значений. Какой вывод можно сделать исходя из найденных значений?
	\item Используйте критерий перестановок. Каково в этом случае значение $\pvalue$. Какой вывод можно сделать?
\end{itemize}


\subsubsection*{Задача 12 [2 балла]}
Маршрут грузового состава начинается в пункте $A$ и последовательно проходит через пункты $B_0$, $B_1$ и т.д. По прибытии в очередной пункт те составы, которые направлялись в этот пункт, отцепляются. Очередной состав из 500 грузовых вагонов отправился из пункта $A$ вдоль пунктов $B_0$, $B_1$, \dots. В таблице приведено количество отцепленных составов в каждом из пунктов (последним пунктом в данном случае оказался пункт $B_9$).

\begin{table}[!h]
	\centering
	\begin{tabular}{|l|c|c|c|c|c|c|c|c|c|c|}
		\hline
		Пункт & $B_0$  & $B_1$  & $B_2$   & $B_3$   & $B_4$   & $B_5$  & $B_6$  & $B_7$ & $B_8$ & $B_9$ \\ \hline
		Количество составов & 15 & 55 & 126 & 110 & 113 & 49 & 20 & 9 & 2 & 1 \\
		\hline
	\end{tabular}
\end{table}

Возникло предположение, что распределение грузовых составов по пунктам назначения можно описать некоторым дискретным распределением, где $P(X = B_i)$ --- вероятность того, что состав направляется в пункт $B_i$. В рамках данного предположения требуется провести проверку следующих гипотез на уровне значимости $\alpha = 0.05$ и найти p-value:
\begin{enumerate}
	\item $\boldX \sim \Poisson(\theta)$, т.е. $P(X = B_j) = e^{-\theta} \frac{\theta^j}{j!}$, где $j \ge 0$.
	\item $\boldX \sim \Binomial(m, p)$, т.е. $P(X = B_j) = C^j_m p^j (1-p)^{m - j}$, где $j \in \{0, \dots, 9\}$ и $m = 9$.
\end{enumerate}

\textit{Подсказка. Воспользуйтесь параметрическим критерием хи-квадрат.}

\end{document}